\documentclass[10pt, letterpaper]{article}

\usepackage[
    ignoreheadfoot, % set margins without considering header and footer
    top=2 cm, % seperation between body and page edge from the top
    bottom=2 cm, % seperation between body and page edge from the bottom
    left=2 cm, % seperation between body and page edge from the left
    right=2 cm, % seperation between body and page edge from the right
    footskip=1.0 cm, % seperation between body and footer
    % showframe % for debugging 
]{geometry} % for adjusting page geometry
\usepackage{titlesec} % for customizing section titles
\usepackage{tabularx} % for making tables with fixed width columns
\usepackage{array} % tabularx requires this
\usepackage[dvipsnames]{xcolor} % for coloring text
\definecolor{primaryColor}{RGB}{0, 0, 0} % define primary color
\usepackage{enumitem} % for customizing lists
\usepackage{fontawesome5} % for using icons
\usepackage{amsmath} % for math
\usepackage[
    pdftitle={John Doe's CV},
    pdfauthor={John Doe},
    pdfcreator={LaTeX with RenderCV},
    colorlinks=true,
    urlcolor=primaryColor
]{hyperref} % for links, metadata and bookmarks
\usepackage[pscoord]{eso-pic} % for floating text on the page
\usepackage{calc} % for calculating lengths
\usepackage{bookmark} % for bookmarks
\usepackage{lastpage} % for getting the total number of pages
\usepackage{changepage} % for one column entries (adjustwidth environment)
\usepackage{paracol} % for two and three column entries
\usepackage{ifthen} % for conditional statements
\usepackage{needspace} % for avoiding page brake right after the section title
\usepackage{iftex} % check if engine is pdflatex, xetex or luatex

% Ensure that generate pdf is machine readable/ATS parsable:
\ifPDFTeX
    \input{glyphtounicode}
    \pdfgentounicode=1
    \usepackage[T1]{fontenc}
    \usepackage[utf8]{inputenc}
    \usepackage{lmodern}
\fi

\usepackage{charter}
\raggedright
\AtBeginEnvironment{adjustwidth}{\partopsep0pt} % remove space before adjustwidth environment
\pagestyle{empty} % no header or footer
\setcounter{secnumdepth}{0} % no section numbering
\setlength{\parindent}{0pt} % no indentation
\setlength{\topskip}{0pt} % no top skip
\setlength{\columnsep}{0.15cm} % set column seperation
\pagenumbering{gobble} % no page numbering

\titleformat{\section}{\needspace{4\baselineskip}\bfseries\large}{}{0pt}{}[\vspace{1pt}\titlerule]

\titlespacing{\section}{
    % left space:
    -1pt
}{
    % top space:
    0.3 cm
}{
    % bottom space:
    0.2 cm
} % section title spacing

\renewcommand\labelitemi{$\vcenter{\hbox{\small$\bullet$}}$} % custom bullet points
\newenvironment{highlights}{
    \begin{itemize}[
        topsep=0.10 cm,
        parsep=0.10 cm,
        partopsep=0pt,
        itemsep=0pt,
        leftmargin=0 cm + 10pt
    ]
}{
    \end{itemize}
} % new environment for highlights


\newenvironment{highlightsforbulletentries}{
    \begin{itemize}[
        topsep=0.10 cm,
        parsep=0.10 cm,
        partopsep=0pt,
        itemsep=0pt,
        leftmargin=10pt
    ]
}{
    \end{itemize}
} % new environment for highlights for bullet entries

\newenvironment{onecolentry}{
    \begin{adjustwidth}{
        0 cm + 0.00001 cm
    }{
        0 cm + 0.00001 cm
    }
}{
    \end{adjustwidth}
} % new environment for one column entries

\newenvironment{twocolentry}[2][]{
    \onecolentry
    \def\secondColumn{#2}
    \setcolumnwidth{\fill, 4.5 cm}
    \begin{paracol}{2}
}{
    \switchcolumn \raggedleft \secondColumn
    \end{paracol}
    \endonecolentry
} % new environment for two column entries

\newenvironment{threecolentry}[3][]{
    \onecolentry
    \def\thirdColumn{#3}
    \setcolumnwidth{, \fill, 4.5 cm}
    \begin{paracol}{3}
    {\raggedright #2} \switchcolumn
}{
    \switchcolumn \raggedleft \thirdColumn
    \end{paracol}
    \endonecolentry
} % new environment for three column entries

\newenvironment{header}{
    \setlength{\topsep}{0pt}\par\kern\topsep\centering\linespread{1.5}
}{
    \par\kern\topsep
} % new environment for the header

\newcommand{\placelastupdatedtext}{% \placetextbox{<horizontal pos>}{<vertical pos>}{<stuff>}
  \AddToShipoutPictureFG*{% Add <stuff> to current page foreground
    \put(
        \LenToUnit{\paperwidth-2 cm-0 cm+0.05cm},
        \LenToUnit{\paperheight-1.0 cm}
    ){\vtop{{\null}\makebox[0pt][c]{
        \small\color{gray}\textit{Last updated in July 2024}\hspace{\widthof{Last updated in July 2024}}
    }}}%
  }%
}%

% save the original href command in a new command:
\let\hrefWithoutArrow\href



\begin{document}
    \newcommand{\AND}{\unskip
        \cleaders\copy\ANDbox\hskip\wd\ANDbox
        \ignorespaces
    }
    \newsavebox\ANDbox
    \sbox\ANDbox{$|$}

    \begin{header}
        \fontsize{25 pt}{25 pt}\selectfont Bruno Jardim

        \vspace{5 pt}

        \normalsize
        \mbox{Braga, Portugal}%
        \kern 5.0 pt%
        \AND%
        \kern 5.0 pt%
        \mbox{\hrefWithoutArrow{mailto:brunofjm@gmail.com}{brunofjm@gmail.com}}%
        \kern 5.0 pt%
        \AND%
        \kern 5.0 pt%
        \mbox{\hrefWithoutArrow{tel:+351910871693}{+351910871693}}%
        \kern 5.0 pt%
        \AND%
        \kern 5.0 pt%
        \mbox{\hrefWithoutArrow{https://linkedin.com/in/brunojardim13}{linkedin.com/in/brunojardim13}}%
        \kern 5.0 pt%
        \AND%
        \kern 5.0 pt%
        \mbox{\hrefWithoutArrow{https://github.com/StrillX}{github.com/StrillX}}%
    \end{header}

    \vspace{5 pt - 0.3 cm}

    \section{Education}



        
        \begin{twocolentry}{
            Sept 2019 – July 2022
        }
            \textbf{Universidade do Minho}, BS in Computer Science\end{twocolentry}

        \vspace{0.10 cm}
        \begin{onecolentry}
            \begin{highlights}
                \item \textbf{Coursework:} Computational Theory, Complexity Theory, Algorithms and Data structures, Algorithm Correction, SAT solving
                \item \textbf{Final Project:} A generic tool for verifying safety properties in a First Order Transition Systems
            \end{highlights}
        \end{onecolentry}

        \vspace{0.2cm}

        \begin{twocolentry}{
            Sept 2022 – Oct 2024 (est.)
        }
            \textbf{Universidade do Minho}, MS in Formal Methods and Cryptography\end{twocolentry}
        Informatics Engineering (Master)
        \vspace{0.10 cm}
        \begin{onecolentry}
            \begin{highlights}
                \item \textbf{Coursework (Formal Methods):} Algorithm and Program Verification, Distributed Algorithms Verification, Cyber-Physical Programming, Program Design By Calculation
                \item \textbf{Coursework (Cryptography):} Cryptographic Structures, Post-Quantum Cryptography, Vulnerability Detection/Exploitation, Linux Security, Security Engineering
                \item \textbf{Dissertation:} Analyzing quantum learning protocols with ZX
                
                %\small ZX-calculus and variants (ZH,ZW,ZXW,...), Quantum Machine Learning, Quantum Walks, Measurement Based Quantum Computing
                
            \end{highlights}
        \end{onecolentry}



    
    \section{Experience}


        \begin{twocolentry}{
            August 2024 – (ongoing)
        }
            \textbf{Assistant Researcher}, HASLab/INESCTEC -- Universidade do Minho, Braga, PT\end{twocolentry}
        Research Grant

        \vspace{0.10 cm}
        \vspace{0.2 cm}

        \begin{twocolentry}{
            June 2023 – Dec 2023
        }
            \textbf{Full-stack Developer}, Micro-net -- Braga, PT\end{twocolentry}
        Part-time
        \vspace{0.10 cm}
        \begin{onecolentry}
            \begin{highlights}
                \item Developed a hotel self check-in solution
                \item Ported a billing/management desktop application to a web application 
            \end{highlights}
        \end{onecolentry}

        \vspace{0.10 cm}
        \vspace{0.2 cm}
            
        \begin{twocolentry}{
            July 2022 – Sept 2022
        }
            \textbf{Data Analyst}, Checkmarx -- Braga, PT\end{twocolentry}
        Summer Internship
        

        

    
    \section{Publications}



        
        \begin{samepage}
            \begin{twocolentry}{
                Jan 2024
            }
                \textbf{I/O Behaviour Analysis on Android Targeted Ransomware}
            \end{twocolentry}

            \vspace{0.10 cm}
            
            \begin{onecolentry}
                \mbox{Beatriz Oliveira}, \mbox{\textbf{\textit{Bruno Jardim}}}, \mbox{Bruno Pereira}

                \vspace{0.10 cm}
            Unpublished
        \end{onecolentry}
        \end{samepage}

        \vspace{0.2 cm}

        \begin{samepage}
            \begin{twocolentry}{
                November 2024
            }
                \textbf{Reconfiguring staggered quantum walks with ZX}
            \end{twocolentry}

            \vspace{0.10 cm}
            
            \begin{onecolentry}
                \mbox{\textbf{\textit{Bruno Jardim}}}, \mbox{Jaime Santos}, \mbox{Luís S. Barbosa}

                \vspace{0.10 cm}
            Publisehd in the ReacTS'24 Workshop
        \end{onecolentry}
        \end{samepage}


    
    \section{Projects}



        
        \begin{twocolentry}{
            \href{https://github.com/Alef-Keuffer/FOTS-Prover}{github.com/Alef-Keuffer/FOTS-Prover}
        }
            \textbf{A generic tool for verifying safety properties in a First Order Transition Systems}\end{twocolentry}

        \vspace{0.10 cm}
        \begin{onecolentry}
            \begin{highlights}
                \item Implemented 4 different property verification techniques, those being: Bounded Model Checking, K-induction, Interpolant Model Checking and Property Directed Reachability. These techniques can be then used to verify safety properties on First Order Transition Systems
                \item Tools Used: Python, PySMT
            \end{highlights}
        \end{onecolentry}


        \vspace{0.2 cm}

        \begin{twocolentry}{
            \href{https://github.com/StrillX/EC}{github.com/StrillX/EC}
        }
            \textbf{Cryptographic Structures}\end{twocolentry}

        \vspace{0.10 cm}
        \begin{onecolentry}
            \begin{highlights}
                \item Diverse implementations of different cryptographic protocols, both pre and post quatum.
                \item Tools Used: Python, Sagemath
            \end{highlights}
        \end{onecolentry}


    
    \section{Additional Experience}



        
        \begin{onecolentry}
            \textbf{Mentor at Coderdojo (2022-2023):} Taught children between the ages of 7 and 17 how to program.
        \end{onecolentry}
    
        \vspace{0.2 cm}

        \begin{onecolentry}
            \textbf{Organizer of JOIN22 \& JOIN23:} An event with pertinent topics in the current landscape of CS and Software Engineering, sponsored by a diverse set of companies in those respective fields.
        \end{onecolentry}
        

    \section{Technologies \& Other}

        \begin{onecolentry}
            \textbf{Concepts \& Tools:} ZX-calculus and variants (ZH, ZW, ZXW, ...), Quantum Computing, Measurement Based Quantum Computing, Quantum Machine Learning, TLA+, Frama-C, CBMC, Alloy
        \end{onecolentry}

        \vspace{0.2 cm}

        \begin{onecolentry}
            \textbf{Languages:} Python, C++, C, Java, SQL, JavaScript, NextJS, Haskell, Erlang
        \end{onecolentry}

        \vspace{0.2 cm}

        \begin{onecolentry}
            \textbf{Software:} Linux, Git, Bash
        \end{onecolentry}



    

\end{document}